%%--------------------------------------------------------
%% example.tex
%%
%%--------------------------------------------------------
\documentclass[12pt]{article}

%%%%%%%%%%%%%%%%%%%%%%%%%%%%%%%%%%%%%%%%%%%%%%%%%%%%%%%%%%%%%%%%%%
%%% Comment out to use biblatex instead of bibtex
%%%%%%%%%%%%%%%%%%%%%%%%%%%%%%%%%%%%%%%%%%%%%%%%%%%%%%%%%%%%%%%%%%
\def\UseBibLatex{1}

%%%%%%%%%%%%%%%%%%%%%%%%%%%%%%%%%%%%%%%%%%%%%%%%%%%%%%%%%%%%%%%%%
% Put all your private style files/class style files in the styles/
% subdirectory. The following command guarantee that latex would find
% it.
%%%%%%%%%%%%%%%%%%%%%%%%%%%%%%%%%%%%%%%%%%%%%%%%%%%%%%%%%%%%%%%%%

\makeatletter
\def\input@path{{styles/}}
\makeatother


%%%%%%%%%%%%%%%%%%%%%%%%%%%%%%%%%%%%%%%%%%%%%%%%%%%%%%%%%%%%%%%%%%
% A modified usepackge command that checks for style files in the
% styles/ subdirectory.
%%%%%%%%%%%%%%%%%%%%%%%%%%%%%%%%%%%%%%%%%%%%%%%%%%%%%%%%%%%%%%%%%% 
\newcommand{\UsePackage}[1]{%
  \IfFileExists{styles/#1.sty}{%
      \usepackage{styles/#1}%
   }{%
      \IfFileExists{../styles/#1.sty}{%
         \usepackage{../styles/#1}%
      }{%
         \usepackage{#1}%
      }%
   }%
}


\usepackage[T1]{fontenc}
\usepackage{lmodern}
\usepackage{textcomp}
\usepackage[utf8]{inputenc}
\usepackage[UTF8]{ctex}
\usepackage{amsmath}%
\usepackage{amssymb}%
\usepackage[table]{xcolor}%

\setlength{\marginparwidth}{6cm} 
\usepackage{todonotes}
\usepackage[in]{fullpage}%

\usepackage[amsmath,thmmarks]{ntheorem}%
\theoremseparator{.}%

\usepackage{titlesec}%
\titlelabel{\thetitle. }%
\usepackage{xcolor}%
\usepackage{mleftright}%
\usepackage{xspace}%
\usepackage{graphicx}
\usepackage{hyperref}%

\newcommand{\hrefb}[3][black]{\href{#2}{\color{#1}{#3}}}%

\usepackage{hyperref}%
\hypersetup{%
      unicode,
      breaklinks,%
      colorlinks=true,%
      urlcolor=[rgb]{0.25,0.0,0.0},%
      linkcolor=[rgb]{0.5,0.0,0.0},%
      citecolor=[rgb]{0,0.2,0.445},%
      filecolor=[rgb]{0,0,0.4},
      anchorcolor=[rgb]={0.0,0.1,0.2}%
}
\usepackage[ocgcolorlinks]{ocgx2}

%%%%%%%%%%%%%%%%%%%%%%%%%%%%%%%%%%%%%%%%%%%%%%%%%%%%%%%%%%%%%%%%%%%%%%%%
% Defining theorem like environments
%

\theoremseparator{.}%

\theoremstyle{plain}%
\newtheorem{theorem}{Theorem}[section]

\newtheorem{lemma}[theorem]{Lemma}
\newtheorem{conjecture}[theorem]{Conjecture}
\newtheorem{corollary}[theorem]{Corollary}
\newtheorem{claim}[theorem]{Claim}%
\newtheorem{fact}[theorem]{Fact}
\newtheorem{observation}[theorem]{Observation}
\newtheorem{invariant}[theorem]{Invariant}
\newtheorem{question}[theorem]{Question}
\newtheorem{proposition}[theorem]{Proposition}
\newtheorem{prop}[theorem]{Proposition}
\newtheorem{openproblem}[theorem]{Open Problem}

\theoremstyle{plain}%
\theoremheaderfont{\sf} \theorembodyfont{\upshape}%
\newtheorem*{remark:unnumbered}[theorem]{Remark}%
\newtheorem*{remarks}[theorem]{Remarks}%
\newtheorem{remark}[theorem]{Remark}%
\newtheorem{definition}[theorem]{Definition}
\newtheorem{defn}[theorem]{Definition}
\newtheorem{example}[theorem]{Example}
\newtheorem{exercise}[theorem]{Exercise}
\newtheorem{problem}[theorem]{Problem}
\newtheorem{xca}[theorem]{Exercise}
\newtheorem{exercise_h}[theorem]{Exercise}
\newtheorem{assumption}[theorem]{Assumption}%

% Proof environment
\newcommand{\myqedsymbol}{\rule{2mm}{2mm}}

\theoremheaderfont{\em}%
\theorembodyfont{\upshape}%
\theoremstyle{nonumberplain}%
\theoremseparator{}%
\theoremsymbol{\myqedsymbol}%
\newtheorem{proof}{Proof:}%

\newtheorem{proofof}{Proof of\!}%

% theorem block end
%%%%%%%%%%%%%%%%%%%%%%%%%%%%%%%%%%%%%%%%%%%%%%%%%%%%%%%%%%%%%%%%%%%%


%%%%%%%%%%%%%%%%%%%%%%%%%%%%%%%%%%%%%%%%%%%%%%%%%%%%%%%%%%%%%%%%%% 5
% Color emph

\providecommand{\emphind}[1]{}%
\renewcommand{\emphind}[1]{\emph{#1}\index{#1}}

\definecolor{blue25emph}{rgb}{0, 0, 11}

\providecommand{\emphic}[2]{}
\renewcommand{\emphic}[2]{\textcolor{blue25emph}{%
      \textbf{\emph{#1}}}\index{#2}}

\providecommand{\emphi}[1]{}%
\renewcommand{\emphi}[1]{\emphic{#1}{#1}}

\definecolor{almostblack}{rgb}{0, 0, 0.3}

\providecommand{\emphw}[1]{}%
\renewcommand{\emphw}[1]{{\textcolor{almostblack}{\emph{#1}}}}%

\providecommand{\emphOnly}[1]{}%
\renewcommand{\emphOnly}[1]{\emph{\textcolor{blue25}{\textbf{#1}}}}

% Color emph - end
%%%%%%%%%%%%%%%%%%%%%%%%%%%%%%%%%%%%%%%%%%%%%%%%%%%%%%%%%%%%%%%%%% 5

%%%%%%%%%%%%%%%%%%%%%%%%%%%%%%%%%%%%%%%%%%%%%%%%%%%%%%%%%%%%%%%%%%%
% Authors thanks
%%%%%%%%%%%%%%%%%%%%%%%%%%%%%%%%%%%%%%%%%%%%%%%%%%%%%%%%%%%%%%%%%%%

\newcommand{\Thanks}[1]{%
   \thanks{%
      Department of Mathematics; %
      University of Michigan; %
      \href{rynnefan@umich.edu}{rynnefan@umich.edu}; %
   #1%
   }%
}

%%%%%%%%%%%%%%%%%%%%%%%%%%%%%%%%%%%%%%%%%%%%%%%%%
\newcommand{\fql}[1]{%   
\todo[author=Qiulin,inline,color=blue!25]{#1}}
\newcommand{\ycy}[1]{%   
\todo[author=Chunye,inline,color=red!25]{#1}}


%%%%%%%%%%%%%%%%%%%%%%%%%%%%%%%%%%%%%%%%%%%%%%%%%%%%%%%%%%%%%%%%%%%%%%
%    Handling references
%%%%%%%%%%%%%%%%%%%%%%%%%%%%%%%%%%%%%%%%%%%%%%%%%%%%%%%%%%%%%%%%%%%%%%

\newcommand{\HLink}[2]{\hyperref[#2]{#1~\ref*{#2}}}
\newcommand{\HLinkSuffix}[3]{\hyperref[#2]{#1\ref*{#2}{#3}}}

\newcommand{\figlab}[1]{\label{fig:#1}}
\newcommand{\figref}[1]{\HLink{Figure}{fig:#1}}

\newcommand{\thmlab}[1]{{\label{theo:#1}}}
\newcommand{\thmref}[1]{\HLink{Theorem}{theo:#1}}

\newcommand{\remlab}[1]{\label{rem:#1}}
\newcommand{\remref}[1]{\HLink{Remark}{rem:#1}}%

\newcommand{\corlab}[1]{\label{cor:#1}}
\newcommand{\corref}[1]{\HLink{Corollary}{cor:#1}}%

\providecommand{\deflab}[1]{}
\renewcommand{\deflab}[1]{\label{def:#1}}
\newcommand{\defref}[1]{\HLink{Definition}{def:#1}}

\newcommand{\lemlab}[1]{\label{lemma:#1}}
\newcommand{\lemref}[1]{\HLink{Lemma}{lemma:#1}}%

\providecommand{\eqlab}[1]{}%
\renewcommand{\eqlab}[1]{\label{equation:#1}}
\newcommand{\Eqref}[1]{\HLinkSuffix{Eq.~(}{equation:#1}{)}}

%%%%%%%%%%%%%%%%%%%%%%%%%%%%%%%%%%%%%%%%%%%%%%%%%%%%%%%%%%%%%%%%%%%

\newcommand{\remove}[1]{}%
\newcommand{\Set}[2]{\left\{ #1 \;\middle\vert\; #2 \right\}}

\newcommand{\pth}[1]{\mleft(#1\mright)}%

\newcommand{\ProbC}{{\mathbb{P}}}
\newcommand{\ExC}{{\mathbb{E}}}
\newcommand{\VarC}{{\mathbb{V}}}

\newcommand{\Prob}[1]{\ProbC\mleft[ #1 \mright]}
\newcommand{\Ex}[1]{\ExC\mleft[ #1 \mright]}
\newcommand{\Var}[1]{\VarC\mleft[ #1 \mright]}


\newcommand{\ceil}[1]{\mleft\lceil {#1} \mright\rceil}
\newcommand{\floor}[1]{\mleft\lfloor {#1} \mright\rfloor}

\newcommand{\brc}[1]{\left\{ {#1} \right\}}
\newcommand{\set}[1]{\brc{#1}}%

\newcommand{\cardin}[1]{\left\lvert {#1} \right\rvert}%

\renewcommand{\th}{th\xspace}
\newcommand{\ds}{\displaystyle}%

\renewcommand{\Re}{\mathbb{R}}%
\newcommand{\reals}{\Re}%


%%%%%%%%%%%%%%%%%%%%%%%%%%%%%%%%%%%%%%%%%%%%%%%%%%%%%%%%%%%%%%%%%%%%%%%%%
% Defining comptenum environment using enumitem
\usepackage[inline]{enumitem}

\newlist{compactenumA}{enumerate}{5}%
\setlist[compactenumA]{topsep=0pt,itemsep=-1ex,partopsep=1ex,parsep=1ex,%
   label=(\Alph*)}%

\newlist{compactenuma}{enumerate}{5}%
\setlist[compactenuma]{topsep=0pt,itemsep=-1ex,partopsep=1ex,parsep=1ex,%
   label=(\alph*)}%

\newlist{compactenumI}{enumerate}{5}%
\setlist[compactenumI]{topsep=0pt,itemsep=-1ex,partopsep=1ex,parsep=1ex,%
   label=(\Roman*)}%

\newlist{compactenumi}{enumerate}{5}%
\setlist[compactenumi]{topsep=0pt,itemsep=-1ex,partopsep=1ex,parsep=1ex,%
   label=(\roman*)}%

\newlist{compactitem}{itemize}{5}%
\setlist[compactitem]{topsep=0pt,itemsep=-1ex,partopsep=1ex,parsep=1ex,%
   label=\ensuremath{\bullet}}%


%%%%%%%%%%%%%%%%%%%%%%%%%%%%%%%%%%%%%%%%%%%%%%%%%%%%%%%%%%%%%%%%%%%%%%%%%%

%%%%%%%%%%%%%%%%%%%%%%%%%%%%%%%%%%%%%%%%%%%%%%%%%%%%%%%%%%%%%%%%%%%
% Biblatex....
%
\providecommand{\BibLatexMode}[1]{}
\providecommand{\BibTexMode}[1]{}

\ifx\UseBibLatex\undefined%
  \renewcommand{\BibLatexMode}[1]{}
  \renewcommand{\BibTexMode}[1]{#1}
\else
  \renewcommand{\BibLatexMode}[1]{#1}
  \renewcommand{\BibTexMode}[1]{}
\fi


% Bib latex stuff
\BibLatexMode{%
   \usepackage[bibencoding=utf8,style=alphabetic,backend=biber]{biblatex}%
   \UsePackage{my_biblatex}%
}

%
%%%%%%%%%%%%%%%%%%%%%%%%%%%%%%%%%%%%%%%%%%%%%%%%%%%%%%%%%%%%%%%%%%%

\numberwithin{figure}{section}%
\numberwithin{table}{section}%
\numberwithin{equation}{section}%



%%%%%%%%%%%%%%%%%%%%%%%%%%%%%%%%%%%%%%%%%%%%%%%%%%%%%%%%%%%%%%%%%%%
%%%%%%%%%%%%%%%%%%%%%%%%%%%%%%%%%%%%%%%%%%%%%%%%%%%%%%%%%%%%%%%%%%%
% Papers specific commands...
%%%%%%%%%%%%%%%%%%%%%%%%%%%%%%%%%%%%%%%%%%%%%%%%%%%%%%%%
%%%%%%%%%%%%%%%%%%%%%%%%%%%%%%%%%%%%%%%%%%%%%%%%%%%%%%%%



%%%%%%%%%%%%%%%%%%%%%%%%%%%%%%%%%%%%%%%%%%%%%%%%%%%%%%%%
%%BeginIpePreamble
%%%%%%%%%%%%%%%%%%%%%%%%%%%%%%%%%%%%%%%%%%%%%%%%%%%%%%%%


%%%%%%%%%%%%%%%%%%%%%%%%%%%%%%%%%%%%%%%%%%%%%%%%%%%%%%%%
%%EndIpePreamble
%%%%%%%%%%%%%%%%%%%%%%%%%%%%%%%%%%%%%%%%%%%%%%%%%%%%%%%%
%
\BibLatexMode{%
   \bibliography{template}
}

\begin{document}

\title{Math 454: Boundary problems for Partial Differential Equations}

\author{%
   Qiulin Fan%
}

\date{2024/9}

\maketitle

%%%%%%%%%%%%%%%%%%%%%%%%%%%%%%%%%%%%%%%%%%%%%%%%%%%%%%
%%%%%%%%%%%%%%%%%%%%%%%%%%%%%%%%%%%%%%%%%%%%%%%%%%%%%%
\section*{STYLE}
\subsection{A subsection}
以下为本文件的 style。如果您正阅读此笔记请忽略它,因为这是作者防止自己忘记设置的,,
\begin{defn}
\deflab{FQL}%
FQL
\\ \emphi{FQL} 是此 note 的作者.
\end{defn}
\begin{lemma}
    这是一个 Lemma.
\end{lemma}
\begin{theorem}
\thmlab{FQL's Theorem}
FQL's Theorem.
\\  这是一个 Theorem.
\end{theorem}
\begin{proof}
    无需证明.
\end{proof}
\begin{equation}
    \frac{\partial{u}}{\partial{x_1}} + \frac{\partial^2{u}}{\partial{x_2}} = 114514u
\end{equation}
\paragraph{A Paragraph}
这是一个 Paragraph.
\fql{注意到这是显然的.}
\ycy{注意到这是trivial的.}
\defref{FQL} and \thmref{FQL's Theorem} 是这个 Note 中最重要的部分.

以下为一个 CompactEnum.
\begin{compactenumI}
    \item Enum1
    \item Enum2
\end{compactenumI}
继续 Enum:
\begin{compactenumI}[resume]
    \smallskip%
    \item Enum3 continued
\end{compactenumI}



\section{PDE 以及 Linear PDE 的定义}
一个微分方程,不论是 ODE 还是 PDE,求解的是满足某个涉及微分运算的等式的函数的集合。PDE 和 ODE 的区别就在于 ODE 把满足这个等是的函数限制在了一元函数中,因而等式中使用微分算子;而 PDE 则不限制函数的变量数量,可以有任意有限个自变量,因此等式中使用偏微分算子.
\begin{defn}
    \deflab{PDE}
    PDE:
    \\
    一个 PDE 就是如下形式的一个等式,其中的变量为 $u = f(x_1,  \cdots, x_n)$:
    $$
    \\F(u, x_1, x_2, \ldots, x_n, \frac{\partial u}{\partial x_1}, \frac{\partial u}{\partial x_2}, \ldots, \frac{\partial u}{\partial x_n}, \frac{\partial^2 u}{\partial x_1 ^ 2}, \frac{\partial^2 u}{\partial x_1 x_2 }, \cdots) = 0
    $$
    其中,F 是关于(1) $u$,(2) $u$ 的所有自变量,以及(3) $u$ 对任意数量的自变量的任意 order 的 partial derivatives 的函数.
    我们可以把这个 PDE 简写为 $F[u] = 0$
\end{defn}

\begin{defn}
    \deflab{order of PDE}
    一个 PDE 的 order 就是它其中包含的 partial derivatives 的最高 order.
\end{defn}

\begin{defn}
    \deflab{Linear PDE}
    Linear PDE:
    \\
    如果 PDE $F[u] = F(u, x_1, x_2, \ldots, x_n, \frac{\partial u}{\partial x_1}, \frac{\partial u}{\partial x_2}, \ldots, \frac{\partial u}{\partial x_n}, \frac{\partial^2 u}{\partial x_1 ^ 2}, \frac{\partial^2 u}{\partial x_1 x_2 }, \cdots) = 0$ 
    满足:
    \\$F$ 是 $u, \frac{\partial u}{\partial x_1}, \frac{\partial u}{\partial x_2}, \ldots, \frac{\partial u}{\partial x_n}, \frac{\partial^2 u}{\partial x_1 ^ 2}, \frac{\partial^2 u}{\partial x_1 x_2 }, \cdots$ 的 linear function (\textbf{即对未知函数 $u$ 以及 $u$ 的所有偏导数 linear}),那么就称 $F$ 是一个 \textbf{Linear PDE}.
    \\ 也就是说对于任意 $u_1 = f_1(x_1, \cdots, x_n)$ 和 $u_2 = f_2(x_1, \cdots, x_n)$ 以及 const $c$,$$F[u_1 + u_2] = F[u_1] + F[u_2] \; , F[cu] = cF[u]$$
    因而显然 \textbf{一个 Linear PDE 一定是如下形式的一个等式},其中的变量为 $u = f(x_1,  \cdots, c_n)$:
    $$
        \sum_{1 \leq |\alpha| \leq m} c_{\alpha}(x_1, \cdots, x_n) \frac{\partial ^ {|\alpha|} {u}}{\partial{x_1}^{\alpha_1} \cdots \partial{x_n}^{\alpha_n}} = d(x_1, \cdots, x_n)
    $$
    因为 $\forall u_1, u_2 \in C^{|\alpha|}(\mathbf{R}^n)$,
    \begin{align}
         \sum_{1 \leq |\alpha| \leq m} c_{\alpha}(x_1, \cdots, x_n) \frac{\partial ^ {|\alpha|} {(u_1 + u_2)}}{\partial{x_1}^{\alpha_1} \cdots \partial{x_n}^{\alpha_n}}
         &= 2d(x_1, \cdots, x_n) \\
         &= \sum_{1 \leq |\alpha| \leq m} c_{\alpha}(x_1, \cdots, x_n) \frac{\partial ^ {|\alpha|} {u_2}}{\partial{x_1}^{\alpha_1} \cdots \partial{x_n}^{\alpha_n}}\\
   & + \sum_{1 \leq |\alpha| \leq m} c_{\alpha}(x_1, \cdots, x_n) \frac{\partial ^ {|\alpha|} {u_1}}{\partial{x_1}^{\alpha_1} \cdots \partial{x_n}^{\alpha_n}}
    \end{align}
    
我们把所有的 $c_\alpha(x_1, \cdots, x_n)$ 称为 coefficient,而把 $d(x_1, \cdots, x_n)$ 称为 source term.
\end{defn}
\fql{注意到,这里面每一个 $c_\alpha$ 作为偏导数前的系数函数,都是一个自变量是 $x_1, \cdots, x_n$ 的函数,不能依赖 $u$, 否则就不 linear 了. \textbf{并且由于偏微分算子本身是 linear 的,于是这个等式一定是关于 $u$ linear 的,并且其他形式都不是 linear PDE.}}

\begin{defn}
    \deflab{homogeneous}
    Homogeneous:
    \\如果一个 PDE $F[u] = 0 $ 的 source term 是 0,就称它是一个 homogenous 的 PDE.
\end{defn}

\proposition 
\textbf{显然,如果一个 PDE/ODE 是 homogeneous 的,那么他的任意两个解 $u_1, u_2$ 的任意 linear combination 一定也是它的一个解.}



\example 
所有的 second-order linear PDEs 都可以写成以下的形式:
$$
a(x,y)u_{xx} + b(x,y)u_{xy} + c(x,y)u_{yy} + d(x,y)u_x + e(x,y)u_y + f(x,y)u = g(x,y)
$$
$a,b,c,d,e,f$:  coefficients,$g$: source term.

\begin{defn}
    \deflab{Classifications of second-order linear PDEs}
    如果任取 $(x,y)$ 都有 $4a(x,y)c(x,y) - b^2(x,y) >0$, 则称这个二阶线性 PDE 是 elliptic (椭圆)的;
    \\如果任取 $(x,y)$ 都有 $4a(x,y)c(x,y) - b^2(x,y) =0$,则称这个二阶线性 PDE 是 parabolic (抛物)的;
    \\如果任取 $(x,y)$ 都有 $4a(x,y)c(x,y) - b^2(x,y) <0$, 则称这个二阶线性 PDE 是 hyperbolic (双曲)的.
\end{defn}


\section{ODE 的解法 review}
\begin{defn}
    \deflab{first-order linear ODE}
    first-order linear ODE:\\
    $$
    y' + p(x)y = q(x)
    $$
    这是 first order linear ODE 的 general form,比 PDE 简单很多.
\end{defn}

\subsection{first-order linear ODE 的解法:}
\paragraph{对于 homogeneous 的: }
1-order linear ODE 如果是 homogeneous 的,那么就是 separable 的。可以通过分离变量来解出:$y(x) = Ce^{-\int p(x)dx}$ (此处严谨意义上我们可以把 $dx, dy$ 作为 differential form 来分开.)

\paragraph{对于 non-homogenous 的:}
我们有一种巧妙的方法来解出:
\begin{compactenumI}
    \item 对于待求解的微分方程 $y' + p(x)y = q(x)$, 我们先解一个 induced 的 homogeneous  的微分方程 $y' - p(x)y = 0$,并任取一个这个新方程的解函数,记作 $\hat{y}$,通常直接把 $C$ 去掉就可以($\hat{y} = \; (= e^{\int p(x) dx})$)
    \item 可以发现,$(\hat{y})' = p(x)y$
    \item 在原微分方程的左右两边乘上 $\hat{y}$,得到 $(\hat{y}y)' = \hat{y}q(x)$
    \item $y = \frac{1}{\hat{y}} (\int \hat{y}q(x)dx + C)$
\end{compactenumI}

\subsection{second-order linear ODE 的解法:}
\paragraph{对于 constant coeff and homogeneous 的}: 形式为 $ay'' + by' + cy = 0$,有一个公式的解法:
\\先解决 characteristic equation $a\lambda^2 + b\lambda + c = 0$ 得到两个解 $\lambda_1, \lambda_2$.
\begin{compactenumI}
    \item 如果 $\lambda_1 \not= \lambda_2$,那么 general solution 就是 $y(x) = C_1 e^{\lambda_1 x} + C_2 e^{\lambda_2 x}$;注意:如果没有实数根,那么这个方程只在 $\mathcal{F}(\mathbb{C}, \mathbb{C})$ 上有解,但是解的形式也一样.
    \item 如果 $\lambda_1 = \lambda_2$,那么 general solution 就是 $y(x) = (C_1 + C_2 x) e^{\lambda x}$
\end{compactenumI}

\remark 
对于 n 阶 ODE,我们需要解一个 n 次的 characteristic equation,通常需要通过 n 个 linearly independent 的齐次解的 linear combination 来描述整个齐次解空间。原因见 ODE 的自学笔记。

\paragraph{对于 constant coeff and non-homogeneous 的}: 
微分方程的形式即我们把 $F[u] = 0 $,其中 $F[u] = L[u] - f(x)$, $f(x)$ 为 source term. 
\\对于每一个 non-homogeneous 的微分方程 $L[u] = f(x)$,我们都可以找到它的 homogeneous 的版本 $L[u] = 0$,并把 $L[u] = 0$ 的 soltion 称为 \textbf{general solution}.\\
下面我们要说一个定理:
\begin{theorem}
    对于任意阶 linear ODE $L[u] = f(x)$:如果 $y_p$ 是 $L[u] = f(x)$ 的一个 solution,而 $y_h$ 是 $L[u] = 0$ 的 general solution,那么 $y_p + y_h$ 就是 $L[u] = f(x)$ 的 solution (的所有形式).
\end{theorem}
\fql{这是显然的。因为 L 是 linear 的,$L[y_p + y_h] = L[y_h] + L[y_p] = 0 + f(x) = f(x)$} 
我们可以先通过上面的方法把二阶线性 homogeneous 方程 $L[u] = 0$ 给解出来.\\
然后,我们只要找到 $L[x] = f(x)$ 的任意一个解,称之为一个 \textbf{particular solution},就可以求出它的整个解空间。
\\找出一个特定解也是有固定的公式方法的。\\
1. 待定系数法 (method of undetermined coefficients): 适用当 source term $f(x)$ 是一个 polynomial, exponential 或者 trigonometric 函数的时候:我们知道方程的解也是一个形式类似形式的函数。比如 $y'' - 3y' + 2y = e^{2x}$,我们可以设 $y_p(x) = Ae^{2x}$,解出 $A$ 就好。\\
2. 变量变换法 (variation of parameters): 适用于任何情况。
公式步骤:
\begin{compactenumI}
    \item general solution 的形式一定是 $y_h(x) = C_1 y_1(x) + C_2 y_2(x)$. 现在我们忽略 $C_1, C_2$, 取出 $y_1(x), y_2(x)$.
    \item 计算 \textbf{Wronskian}: $$W(y_1, y_2) = \begin{vmatrix} y_1(x) & y_2(x) \\ y_1'(x) & y_2'((x) \end{vmatrix} = y_1y_2' - y_2y_1'$$
    \item $$y_p(x) = -y_1(x) \int{\frac{y_2(x)f(x)}{W(y_1, y_2)} dx} + y_2(x) \int{\frac{y_1(x)f(x)}{W(y_1, y_2)} dx}$$
\end{compactenumI}
\begin{proof}
假设 particular solution \( y_p(x) \) 的形式为:
\[y_p = u_1(x) y_1(x) + u_2(x) y_2(x)\]

为了简化计算,我们施加一个条件:$u_1' y_1 + u_2' y_2 = 0$
\\于是 \[y_p' = u_1 y_1' + u_2 y_2',
\;\;y_p'' = u_1' y_1' + u_1 y_1'' + u_2' y_2' + u_2 y_2'' = (u_1' y_1' + u_2' y_2') + (u_1 y_1'' + u_2 y_2'')\]

\textbf{代入原方程}
\[(u_1' y_1' + u_2' y_2') + u_1 y_1'' + u_2 y_2'' + p(u_1 y_1' + u_2 y_2') + q(u_1 y_1 + u_2 y_2) = r(x)\]

因为 \( y_1 \) 和 \( y_2 \) 满足 homogeneous 方程,所以:
\[y_i'' + p y_i' + q y_i = 0 (i = 1, 2), \;\;y_i'' = -p y_i' - q y_i\]

代入后得到:
\[\begin{aligned}
& (u_1' y_1' + u_2' y_2') + u_1 (-p y_1' - q y_1) + u_2 (-p y_2' - q y_2) \\
& \quad + p(u_1 y_1' + u_2 y_2') + q(u_1 y_1 + u_2 y_2) = r(x)
\end{aligned}\]

化简后相互抵消,只剩下$u_1' y_1' + u_2' y_2' = r(x)$.

\textbf{结合之前的条件,得到如下方程组:}

\[\begin{bmatrix}y_1 & y_2 \\y_1' & y_2'\end{bmatrix}
\begin{bmatrix}u_1' \\u_2'\end{bmatrix}=
\begin{bmatrix}0 \\r(x)\end{bmatrix}\]

\[W = \begin{vmatrix}y_1 & y_2 \\y_1' & y_2'\end{vmatrix} = y_1 y_2' - y_1' y_2\]
\textbf{利用克莱姆法则求解}
\[u_1' = \frac{\begin{vmatrix}0 & y_2 \\r(x) & y_2'\end{vmatrix}}{W} = \frac{0 \cdot y_2' - y_2 \cdot r(x)}{W} = -\frac{y_2 r(x)}{W}\]
\[u_2' = \frac{\begin{vmatrix}y_1 & 0 \\y_1' & r(x)
\end{vmatrix}}{W} = \frac{y_1 \cdot r(x) - 0 \cdot y_1'}{W} = \frac{y_1 r(x)}{W}\]
\textbf{积分求解 \( u_1(x) \) 和 \( u_2(x) \)}
\[\begin{aligned}
u_1(x) &= -\int \frac{y_2 r(x)}{W} \, dx + C_1 \\
u_2(x) &= \int \frac{y_1 r(x)}{W} \, dx + C_2
\end{aligned}\]
由于只需要一个 particular solution,可以令积分常数为零。
\[y_p = u_1 y_1 + u_2 y_2 = y_1 \left( -\int \frac{y_2 r(x)}{W} \, dx \right) + y_2 \left( \int \frac{y_1 r(x)}{W} \, dx \right)\]
\end{proof}

\paragraph{对于 nonconstant coeff 的}: 
见 ODE 的自学笔记。
具体有以下:\\
1. power series 解法:当 coeff function 在某点表现为 analytic 的时候使用。
\\2. Frobenius 方法:power series 解法的扩展
\\3. Laplace 变换:对于某些类型的 boundary value problem,可以把 diff equation 转化为 algebraic equation.
次数不展开.

\section{Boundary value problems of PDE}
\subsection{Separation of variables}
我们可以先假设这个 PDE 的解是可以分离为一个 $X$ 的函数和一个 $Y$ 的函数的,即 $u(x,y) = X(x)Y(y)$,然后我们把它代入原二阶 PDE 进行尝试,把 PDE 的左右两边分离 $u$ 关于变量 $x$ 和关于变量 $y$ 的 ODE,如果分离不出来就说明无法找到可分离变量的解。
\\
\fql{通常, second-order linear homogenenous PDEs 都具有可分离变量的解! 并且,大部分 second-order linear homogenenous PDEs 的解系中的任意解都是可分离变量的,即\textbf{我们通过分离变量法得出的结果是一个 complete 的 solution set.} 什么样的二阶 PDE 通过 separation of variables 得出的 solution set 是 complete 的? 请见下 criterion:}

\begin{theorem}
    \thmlab{Separable of variables}
    如果对于 PDE $F[u] = 0$ 满足以下条件, 那么这个 PDE 通过 separable of variables 得出的 soluiton set 是 complete 的:\\
    \begin{compactenumI}
    \item $F[u] = 0$ 是 linear homogeneous PDE.
    \item boundary conditions 也是 linear homogeneous 的.
    \item 分离变量后得到的两边 ODE 对应的微分算子是 self-adjoint operator.
    \item 分离变量后得到的两边 ODE 在 Sturm-Liouville eigenvalue problem 中的 eigenfunctions 形成一个 complete orthogonal system.
\end{compactenumI}
\end{theorem}
这下面两个条件是什么意思?等下再说,我们先不管它。我们先分离一个看看是什么结果。

我们分离 Laplace's equation:
$$
\frac{\partial ^2 u}{\partial x ^ 2} + \frac{\partial ^2 u}{\partial y ^ 2} = 0
$$
分离得到:
$$
Y \frac{d^2 X}{dx^2} = -X \frac{d^2 Y}{dy^2} \Rightarrow  \frac{X''}{X} = - \frac{Y''}{Y}
$$

令 $\frac{X''}{X} = - \frac{Y''}{Y} = \lambda$, 称 $\lambda$ 为 \textbf{separation constant},我们得到了两个 separated ODE: $X'' + \lambda x = 0$, $Y'' - \lambda y = 0$.
我们将 $\lambda$ 的值在 $\mathbb{R}$ 上分类讨论,得到了三个解系,将它们合并起来就是用 separation 法得出的整个解系:

\[
u = 
\begin{cases} 
(A_1 x + A_2)(B_1 y + B_2), \\ 
(A_1 e^{ikx} + A_2 e^{-ikx})(B_1 e^{ky} + B_2 e^{-ky}), \\
(A_1 e^{lx} + A_2 e^{-lx})(B_1 e^{ily} + B_2 e^{-ily})
\end{cases}
\]
如果不选择指数作为基解而是选择 sin, cos (for complex) 及 sinh, cosh (for real) 作为等价的基解,那么这个 solution space 就成为以下的形式:
\[
u = 
\begin{cases} 
(A_1 x + A_2)(B_1 y + B_2), \\
(A_1 \cos(kx) + A_2 \sin(kx))(B_1 \cosh(ky) + B_2 \sinh(ky)), \\
(A_1 \cosh(lx) + A_2 \sinh(lx))(B_1 \cos(ly) + B_2 \sin(ly))
\end{cases}
\]


\subsection{Fitting the solution set to boundary conditions}
\paragraph{} 如果给出 $u(0,y) = 0, u(L,y) = 0, u(x,0) = 0$ 三个 boundary conditions, 那么 $u$ 就被限制到了 $u(x,y) = A \sin{\frac{n \pi x}{L}} sinh{\frac{n \pi x}{L}}, \; n\in \mathbb{N}$ 上;
\\\\但是如果再给出第四个 boundary condition: $u(x, L) = \varphi(x)$,其中 $\varphi(x)$ 是一个函数,那么解空间会是什么样子呢?这个时候单个的 $u(x,y)$ 几乎不可能通过调整系数 $A, B$ 来模拟 $\varphi(x)$ 的行为。(除非 $\varphi(x)$ 正好就是一个正弦函数.)
\\\\这个时候,我们想起来:因为这个微分方程是 linear homogeneous 的,所以任意解的任意 lienar combination 仍然是一个解,可以推广至无穷;但是这有什么用呢?我们需要介绍 Fourier theory: \textbf{满足一定条件的函数,可以用不同频率的 $sin, cos$ 函数来逼近,并且如果在无穷项上收敛到这个函数本身. 即: 通过调整每项系数 $A_i$,可以使得在一定范围内 $\varphi(x) = \sum_{n=1}^{\infty} A_i \sin{\frac{n \pi x}{L}}$}
\\\\也就是说,我们可以调整所有的参数 $A_i$,我们可以让 $u(x,y) = \sum_{n=1}^{\infty} A_i \sin \frac{n \pi x}{L} \sinh \frac{n \pi x}{L}$ 满足这个 boundary condition 的要求. 所以存在一组系数 $\{A_i\}$ 使得 $u(x,y) = \sum_{n=1}^{\infty} A_i \sin \frac{n \pi x}{L} \sinh \frac{n \pi x}{L}$  成为满足 boundary condition 的唯一解。
\\\\这个唯一解是可以求出来的,其中的系数 $A_n = \frac{2}{L \sinh{n \pi }} \int_0^L \varphi(x) \sin \frac{n \pi x}{L}dx$,具体的求法见下一章.



\section{Intro to Fourier Series}

\begin{defn}
    \deflab{Fourier series}
    Fourier series:\\
    一个 Fourier series 就是给定一组系数 $A_0, \{A_n, B_n | n \in \mathbb{N}\}$ 的级数 $$
    A_0 + \sum_{n = 1} ^ \infty (A_n \cos \frac{n \pi x}{L} + B_n \sin \frac{n \pi x}{L}) 
    $$
    我们可以把其中的 $\cos \frac{n \pi x}{L}, \sin \frac{n \pi x}{L}$ 简写为 $\cos_n, \sin_n$,即:
    $$
    A_0 + \sum_{n = 1} ^ \infty (A_n \cos_n + B_n \sin _n)
    $$
\end{defn}

\begin{theorem}
    \thmlab{Represent a function by Fourier series}
    任意一个 real/complex-valued 的函数 $f(x): (-L, L) \rightarrow \mathbb{C}$,只要在 $(-L, L)$ 上满足 \textbf{Dirichlet Conditions},那么就在 $(-L, L)$ 上等于一个Fourier 函数列级数.\\
    即:存在系数 $A_0, \{A_n, B_n | n \in \mathbb{N}\}$ 使得 
     $$
    f(x) = A_0 + \sum_{n = 1} ^ \infty (A_n \cos \frac{n \pi x}{L} + B_n \sin \frac{n \pi x}{L})\; , \; x \in (-L, L)
    $$

    \textbf{Dirichlet Conditions} :$f(x)$ 在 $(-L,L)$ 上的不连续点和极值点有限(不能太不连续,不能太过震荡);Rmk: 在不连续点上傅里叶级数会收敛到左极限和右极限的平均值.
 \end{theorem}



\subsection{Orthogonality of trigonometric function}
\begin{theorem}
    \thmlab{Orthogonality of trigonometric function}
    Orthogonality of trigonometric function:\\
    任取 $n,m \in \mathbb{N}, L > 0$, 都有:
    \[
\int_{-L}^{L} \cos_n \cos_m dx = 
\begin{cases} 
2L, & (n = m = 0), \\ 
L \delta_{nm}, & (\text{otherwise}),
\end{cases}
\]

\[
\int_{-L}^{L} \sin_m \sin_n  dx = 
\begin{cases} 
0, & (n = m = 0), \\ 
L \delta_{nm}, & (\text{otherwise}),
\end{cases}
\]

\[
\int_{-L}^{L} \sin_n \cos_m dx = 0, \quad (\text{for all } n, m).
\]
即:\textbf{在 Hilbert space $L^2([-L, L])$,即 [-L, L] 上的平方可积函数空间上,取内积 $<f(x), g(x)> = \int_{-L}^L f(x)g(x)dx$,任意的 $\sin_m$ 和 $cos_n$ 都是 orthogonal 的},\textbf{而不同频率的 $\cos_n$ 自身之间,以及不同频率 $\sin_n$ 自身之间也是 orthogonal 的}
\end{theorem}


